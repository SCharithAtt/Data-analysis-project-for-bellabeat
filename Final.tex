% Options for packages loaded elsewhere
\PassOptionsToPackage{unicode}{hyperref}
\PassOptionsToPackage{hyphens}{url}
%
\documentclass[
]{article}
\usepackage{amsmath,amssymb}
\usepackage{iftex}
\ifPDFTeX
  \usepackage[T1]{fontenc}
  \usepackage[utf8]{inputenc}
  \usepackage{textcomp} % provide euro and other symbols
\else % if luatex or xetex
  \usepackage{unicode-math} % this also loads fontspec
  \defaultfontfeatures{Scale=MatchLowercase}
  \defaultfontfeatures[\rmfamily]{Ligatures=TeX,Scale=1}
\fi
\usepackage{lmodern}
\ifPDFTeX\else
  % xetex/luatex font selection
\fi
% Use upquote if available, for straight quotes in verbatim environments
\IfFileExists{upquote.sty}{\usepackage{upquote}}{}
\IfFileExists{microtype.sty}{% use microtype if available
  \usepackage[]{microtype}
  \UseMicrotypeSet[protrusion]{basicmath} % disable protrusion for tt fonts
}{}
\makeatletter
\@ifundefined{KOMAClassName}{% if non-KOMA class
  \IfFileExists{parskip.sty}{%
    \usepackage{parskip}
  }{% else
    \setlength{\parindent}{0pt}
    \setlength{\parskip}{6pt plus 2pt minus 1pt}}
}{% if KOMA class
  \KOMAoptions{parskip=half}}
\makeatother
\usepackage{xcolor}
\usepackage[margin=1in]{geometry}
\usepackage{color}
\usepackage{fancyvrb}
\newcommand{\VerbBar}{|}
\newcommand{\VERB}{\Verb[commandchars=\\\{\}]}
\DefineVerbatimEnvironment{Highlighting}{Verbatim}{commandchars=\\\{\}}
% Add ',fontsize=\small' for more characters per line
\usepackage{framed}
\definecolor{shadecolor}{RGB}{248,248,248}
\newenvironment{Shaded}{\begin{snugshade}}{\end{snugshade}}
\newcommand{\AlertTok}[1]{\textcolor[rgb]{0.94,0.16,0.16}{#1}}
\newcommand{\AnnotationTok}[1]{\textcolor[rgb]{0.56,0.35,0.01}{\textbf{\textit{#1}}}}
\newcommand{\AttributeTok}[1]{\textcolor[rgb]{0.13,0.29,0.53}{#1}}
\newcommand{\BaseNTok}[1]{\textcolor[rgb]{0.00,0.00,0.81}{#1}}
\newcommand{\BuiltInTok}[1]{#1}
\newcommand{\CharTok}[1]{\textcolor[rgb]{0.31,0.60,0.02}{#1}}
\newcommand{\CommentTok}[1]{\textcolor[rgb]{0.56,0.35,0.01}{\textit{#1}}}
\newcommand{\CommentVarTok}[1]{\textcolor[rgb]{0.56,0.35,0.01}{\textbf{\textit{#1}}}}
\newcommand{\ConstantTok}[1]{\textcolor[rgb]{0.56,0.35,0.01}{#1}}
\newcommand{\ControlFlowTok}[1]{\textcolor[rgb]{0.13,0.29,0.53}{\textbf{#1}}}
\newcommand{\DataTypeTok}[1]{\textcolor[rgb]{0.13,0.29,0.53}{#1}}
\newcommand{\DecValTok}[1]{\textcolor[rgb]{0.00,0.00,0.81}{#1}}
\newcommand{\DocumentationTok}[1]{\textcolor[rgb]{0.56,0.35,0.01}{\textbf{\textit{#1}}}}
\newcommand{\ErrorTok}[1]{\textcolor[rgb]{0.64,0.00,0.00}{\textbf{#1}}}
\newcommand{\ExtensionTok}[1]{#1}
\newcommand{\FloatTok}[1]{\textcolor[rgb]{0.00,0.00,0.81}{#1}}
\newcommand{\FunctionTok}[1]{\textcolor[rgb]{0.13,0.29,0.53}{\textbf{#1}}}
\newcommand{\ImportTok}[1]{#1}
\newcommand{\InformationTok}[1]{\textcolor[rgb]{0.56,0.35,0.01}{\textbf{\textit{#1}}}}
\newcommand{\KeywordTok}[1]{\textcolor[rgb]{0.13,0.29,0.53}{\textbf{#1}}}
\newcommand{\NormalTok}[1]{#1}
\newcommand{\OperatorTok}[1]{\textcolor[rgb]{0.81,0.36,0.00}{\textbf{#1}}}
\newcommand{\OtherTok}[1]{\textcolor[rgb]{0.56,0.35,0.01}{#1}}
\newcommand{\PreprocessorTok}[1]{\textcolor[rgb]{0.56,0.35,0.01}{\textit{#1}}}
\newcommand{\RegionMarkerTok}[1]{#1}
\newcommand{\SpecialCharTok}[1]{\textcolor[rgb]{0.81,0.36,0.00}{\textbf{#1}}}
\newcommand{\SpecialStringTok}[1]{\textcolor[rgb]{0.31,0.60,0.02}{#1}}
\newcommand{\StringTok}[1]{\textcolor[rgb]{0.31,0.60,0.02}{#1}}
\newcommand{\VariableTok}[1]{\textcolor[rgb]{0.00,0.00,0.00}{#1}}
\newcommand{\VerbatimStringTok}[1]{\textcolor[rgb]{0.31,0.60,0.02}{#1}}
\newcommand{\WarningTok}[1]{\textcolor[rgb]{0.56,0.35,0.01}{\textbf{\textit{#1}}}}
\usepackage{graphicx}
\makeatletter
\def\maxwidth{\ifdim\Gin@nat@width>\linewidth\linewidth\else\Gin@nat@width\fi}
\def\maxheight{\ifdim\Gin@nat@height>\textheight\textheight\else\Gin@nat@height\fi}
\makeatother
% Scale images if necessary, so that they will not overflow the page
% margins by default, and it is still possible to overwrite the defaults
% using explicit options in \includegraphics[width, height, ...]{}
\setkeys{Gin}{width=\maxwidth,height=\maxheight,keepaspectratio}
% Set default figure placement to htbp
\makeatletter
\def\fps@figure{htbp}
\makeatother
\setlength{\emergencystretch}{3em} % prevent overfull lines
\providecommand{\tightlist}{%
  \setlength{\itemsep}{0pt}\setlength{\parskip}{0pt}}
\setcounter{secnumdepth}{-\maxdimen} % remove section numbering
\ifLuaTeX
  \usepackage{selnolig}  % disable illegal ligatures
\fi
\IfFileExists{bookmark.sty}{\usepackage{bookmark}}{\usepackage{hyperref}}
\IfFileExists{xurl.sty}{\usepackage{xurl}}{} % add URL line breaks if available
\urlstyle{same}
\hypersetup{
  pdftitle={How can a fitness technology company play it smart?},
  pdfauthor={Senura Charith Attanayake},
  hidelinks,
  pdfcreator={LaTeX via pandoc}}

\title{How can a fitness technology company play it smart?}
\author{Senura Charith Attanayake}
\date{2024-05-22}

\begin{document}
\maketitle

\hypertarget{business-task}{%
\section{Business Task}\label{business-task}}

Bellabeat, a high-tech manufacturer of health-focused products for
women, wants to unlock new growth opportunities by analyzing smart
device data. The key business task is to analyze usage patterns from
non-Bellabeat smart devices, derive insights on consumer behavior, and
apply these findings to improve Bellabeat's products. The goal is to
inform Bellabeat's marketing strategy with data-driven insights and
provide high-level recommendations to the executive team on how to
optimize their product offerings and market positioning.

\hypertarget{key-questions-driving-the-analysis}{%
\paragraph{Key questions driving the
analysis:}\label{key-questions-driving-the-analysis}}

\begin{enumerate}
\def\labelenumi{\arabic{enumi})}
\tightlist
\item
  What are the emerging trends in smart device usage?
\item
  How can these trends be applied to Bellabeat's customer base?
\item
  What marketing strategies can be recommended based on these trends to
  increase Bellabeat's market presence?
\end{enumerate}

\hypertarget{the-data}{%
\section{The Data}\label{the-data}}

The data used for this analysis comes from a public dataset on smart
device usage, specifically the Fitbit Fitness Tracker data. This dataset
includes daily activity logs, heart rate, sleep monitoring, and other
personal health metrics from 30 Fitbit users who consented to share
their data.

\hypertarget{data-sources}{%
\subsubsection{Data sources:}\label{data-sources}}

Fitbit Fitness Tracker Data: A
\href{https://www.kaggle.com/datasets/arashnic/fitbit}{dataset from
Kaggle} containing detailed information on physical activity, heart
rate, and sleep patterns.

\hypertarget{data-structure}{%
\subsubsection{Data Structure:}\label{data-structure}}

\begin{itemize}
\tightlist
\item
  dailyActivity\_merged.csv: Daily step counts, distance, calories
  burned, and activity levels.
\item
  dailyCalories\_merged.csv: Caloric expenditure data on a daily basis.
\item
  dailySteps\_merged.csv: Number of steps taken per day.
\item
  heartRate\_seconds\_merged.csv: Heart rate data at minute-level
  granularity.
\item
  sleepDay\_merged.csv: Data on sleep duration and sleep efficiency.
\end{itemize}

The dataset allows for detailed exploration of physical activity, sleep
patterns, and heart rate, providing rich insights into how users
interact with their devices and health data.

\hypertarget{limitations}{%
\subsubsection{Limitations:}\label{limitations}}

\textbf{It should be noted that this dataset used has significant
limitations as this was based on only 30 consenting participants. This
is not remotely close to conducting a comprehensive analysis. However,
Based on this data we can get a brief idea about certain patterns and
trends which may serve as a starting point for further assessment.}

\textbf{For certain dataset only about 8 participants provided
comprehensive details, This is not enough to get a thorough idea about
user behavior}

\hypertarget{data-preprocessing}{%
\section{Data preprocessing}\label{data-preprocessing}}

In this preprocessing step, we perform multiple actions to clean and
prepare the Fitbit dataset for analysis. Here's a breakdown of what
happens during this stage:

\hypertarget{library-loading}{%
\subsubsection{Library Loading:}\label{library-loading}}

The tidyverse library is loaded, which includes essential packages for
data manipulation, such as ggplot2, dplyr, and readr.

\hypertarget{data-loading}{%
\subsubsection{Data Loading:}\label{data-loading}}

Various Fitbit CSV datasets are imported using read\_csv(). Each dataset
contains different information related to users' daily activities, heart
rates, sleep patterns, etc. These datasets are assigned to variables
like dailyActivity, dailyCalories, and heartRateSeconds, allowing us to
work with them in subsequent analysis steps. Missing Value Detection:

\begin{itemize}
\tightlist
\item
  The code then checks for missing values across all the imported
  datasets * using colSums(is.na()). This step helps in identifying any
  gaps in the data, which is critical for maintaining data quality in
  the analysis. In particular, the weightLogInfo dataset has missing
  values in the ``Fat'' column, which are addressed by removing the
  entire column using select(-Fat). Exploratory Data Summarization:
\end{itemize}

The skimr and janitor packages are used to generate a clean summary of
the datasets. Functions like skim\_without\_charts() and summary()
provide essential information such as the data's distribution, mean, and
other statistical details for each column. This ensures that we have a
solid understanding of the structure and content of the datasets before
diving into deeper analysis. \#\#\# Summary Statistics:

Summary statistics are printed for each dataset, providing a quick
snapshot of key metrics such as the number of entries, minimum, maximum,
and average values for numeric fields. This helps identify patterns and
potential anomalies in the data. Date and Time Handling:

The lubridate package is used to handle date-time formats. For example,
the Time column in heartRateSeconds is converted to a POSIXct format to
facilitate time-based analysis, enabling accurate plotting of heart rate
over time.

\hypertarget{data-transformation}{%
\subsubsection{Data Transformation:}\label{data-transformation}}

The mutate() function from dplyr is used to calculate new columns based
on existing data. For example, in dailyActivity, the total active
minutes are calculated, and activity levels are converted into
percentages. This transformation makes it easier to interpret how users
spend their time (e.g., sedentary vs.~very active). Data Cleaning:

\begin{itemize}
\tightlist
\item
  Cleaning actions include removing unnecessary columns (like the
  ``Fat'' column from weightLogInfo) and correcting data types (such as
  converting date columns to proper date formats). These steps ensure
  that the data is ready for effective analysis without errors or
  inconsistencies.*
\end{itemize}

\hypertarget{analysis}{%
\section{Analysis}\label{analysis}}

This analysis aims to explore and visualize various aspects of Fitbit
data, including daily activities, heart rates, sleep patterns, and
weight logs. The goal is to derive meaningful insights that can help
users understand their health and activity levels.

\hypertarget{we-start-by-importing-the-libraries}{%
\subsubsection{We start by importing the
Libraries}\label{we-start-by-importing-the-libraries}}

\begin{Shaded}
\begin{Highlighting}[]
\CommentTok{\# Load Required Packages}
\FunctionTok{install.packages}\NormalTok{(}\StringTok{"tidyverse"}\NormalTok{)}
\end{Highlighting}
\end{Shaded}

\begin{verbatim}
## Installing package into 'C:/Users/Senura/AppData/Local/R/win-library/4.3'
## (as 'lib' is unspecified)
\end{verbatim}

\begin{verbatim}
## package 'tidyverse' successfully unpacked and MD5 sums checked
## 
## The downloaded binary packages are in
##  C:\Users\Senura\AppData\Local\Temp\RtmpIf7wpt\downloaded_packages
\end{verbatim}

\begin{Shaded}
\begin{Highlighting}[]
\FunctionTok{library}\NormalTok{(tidyverse)}
\end{Highlighting}
\end{Shaded}

\begin{verbatim}
## Warning: package 'tidyverse' was built under R version 4.3.3
\end{verbatim}

\begin{verbatim}
## -- Attaching core tidyverse packages ------------------------ tidyverse 2.0.0 --
## v dplyr     1.1.4     v readr     2.1.4
## v forcats   1.0.0     v stringr   1.5.1
## v ggplot2   3.4.4     v tibble    3.2.1
## v lubridate 1.9.3     v tidyr     1.3.0
## v purrr     1.0.2
\end{verbatim}

\begin{verbatim}
## -- Conflicts ------------------------------------------ tidyverse_conflicts() --
## x dplyr::filter() masks stats::filter()
## x dplyr::lag()    masks stats::lag()
## i Use the conflicted package (<http://conflicted.r-lib.org/>) to force all conflicts to become errors
\end{verbatim}

\begin{Shaded}
\begin{Highlighting}[]
\FunctionTok{install.packages}\NormalTok{(}\StringTok{"here"}\NormalTok{)}
\end{Highlighting}
\end{Shaded}

\begin{verbatim}
## Installing package into 'C:/Users/Senura/AppData/Local/R/win-library/4.3'
## (as 'lib' is unspecified)
\end{verbatim}

\begin{verbatim}
## package 'here' successfully unpacked and MD5 sums checked
## 
## The downloaded binary packages are in
##  C:\Users\Senura\AppData\Local\Temp\RtmpIf7wpt\downloaded_packages
\end{verbatim}

\begin{Shaded}
\begin{Highlighting}[]
\FunctionTok{library}\NormalTok{(here)}
\end{Highlighting}
\end{Shaded}

\begin{verbatim}
## Warning: package 'here' was built under R version 4.3.3
\end{verbatim}

\begin{verbatim}
## here() starts at C:/Users/Senura/Projects/Bellabeat Data Analysis/Data-analysis-project-for-bellabeat
\end{verbatim}

\begin{Shaded}
\begin{Highlighting}[]
\FunctionTok{install.packages}\NormalTok{(}\StringTok{"skimr"}\NormalTok{)}
\end{Highlighting}
\end{Shaded}

\begin{verbatim}
## Installing package into 'C:/Users/Senura/AppData/Local/R/win-library/4.3'
## (as 'lib' is unspecified)
\end{verbatim}

\begin{verbatim}
## package 'skimr' successfully unpacked and MD5 sums checked
## 
## The downloaded binary packages are in
##  C:\Users\Senura\AppData\Local\Temp\RtmpIf7wpt\downloaded_packages
\end{verbatim}

\begin{Shaded}
\begin{Highlighting}[]
\FunctionTok{library}\NormalTok{(skimr)}
\end{Highlighting}
\end{Shaded}

\begin{verbatim}
## Warning: package 'skimr' was built under R version 4.3.3
\end{verbatim}

\begin{Shaded}
\begin{Highlighting}[]
\FunctionTok{install.packages}\NormalTok{(}\StringTok{"janitor"}\NormalTok{)}
\end{Highlighting}
\end{Shaded}

\begin{verbatim}
## Installing package into 'C:/Users/Senura/AppData/Local/R/win-library/4.3'
## (as 'lib' is unspecified)
\end{verbatim}

\begin{verbatim}
## package 'janitor' successfully unpacked and MD5 sums checked
## 
## The downloaded binary packages are in
##  C:\Users\Senura\AppData\Local\Temp\RtmpIf7wpt\downloaded_packages
\end{verbatim}

\begin{Shaded}
\begin{Highlighting}[]
\FunctionTok{library}\NormalTok{(janitor)}
\end{Highlighting}
\end{Shaded}

\begin{verbatim}
## Warning: package 'janitor' was built under R version 4.3.3
\end{verbatim}

\begin{verbatim}
## 
## Attaching package: 'janitor'
## 
## The following objects are masked from 'package:stats':
## 
##     chisq.test, fisher.test
\end{verbatim}

\hypertarget{each-csv-data-set-is-stored-in-a-variable}{%
\subsubsection{Each CSV data set is stored in a
variable}\label{each-csv-data-set-is-stored-in-a-variable}}

\hypertarget{check-for-missing-values-in-each-dataset}{%
\subsubsection{Check for missing values in each
dataset}\label{check-for-missing-values-in-each-dataset}}

\begin{Shaded}
\begin{Highlighting}[]
\CommentTok{\# Check for missing values}
\NormalTok{missing\_values }\OtherTok{\textless{}{-}} \FunctionTok{lapply}\NormalTok{(}\FunctionTok{list}\NormalTok{(dailyActivity, dailyCalories, dailyIntensities, dailySteps,}
\NormalTok{                               heartRateSeconds, hourlyCalories, hourlyIntensities,}
\NormalTok{                               hourlySteps, minuteCaloriesNarrow, minuteCaloriesWide,}
\NormalTok{                               minuteIntensitiesNarrow, minuteIntensitiesWide,}
\NormalTok{                               minuteMETsNarrow, minuteSleep, minuteStepsNarrow,}
\NormalTok{                               minuteStepsWide, sleepDaily, weightLogInfo), }
                         \ControlFlowTok{function}\NormalTok{(x) }\FunctionTok{colSums}\NormalTok{(}\FunctionTok{is.na}\NormalTok{(x)))}
\end{Highlighting}
\end{Shaded}

\hypertarget{we-look-at-the-summary-statistics-for-each-dataset}{%
\subsubsection{We look at the summary statistics for each
dataset}\label{we-look-at-the-summary-statistics-for-each-dataset}}

\begin{Shaded}
\begin{Highlighting}[]
\CommentTok{\# Check for missing values}
\CommentTok{\# Summary statistics for all datasets}
\FunctionTok{lapply}\NormalTok{(}\FunctionTok{list}\NormalTok{(dailyActivity, dailyCalories, dailyIntensities, dailySteps,}
\NormalTok{            heartRateSeconds, hourlyCalories, hourlyIntensities,}
\NormalTok{            hourlySteps, minuteCaloriesNarrow, minuteCaloriesWide,}
\NormalTok{            minuteIntensitiesNarrow, minuteIntensitiesWide,}
\NormalTok{            minuteMETsNarrow, minuteSleep, minuteStepsNarrow,}
\NormalTok{            minuteStepsWide, sleepDaily, weightLogInfo), summary)}
\end{Highlighting}
\end{Shaded}

\hypertarget{daily-activity-analysis}{%
\subsection{Daily Activity Analysis}\label{daily-activity-analysis}}

\hypertarget{average-calories-burned}{%
\subsubsection{Average Calories Burned}\label{average-calories-burned}}

We calculate the average calories burned per user and visualize the
results.

\includegraphics{Final_files/figure-latex/avgcalories-1.pdf}
\includegraphics{Final_files/figure-latex/avgcalories-2.pdf}
\emph{Explain the significance of calories burned and its implications
for user health.}

\includegraphics{Final_files/figure-latex/activitypercentages -1.pdf}
\emph{Discuss how different activity levels affect overall health and
well-being.}

\hypertarget{weight-log-analysis}{%
\subsection{Weight Log Analysis}\label{weight-log-analysis}}

\hypertarget{bmi-calculation}{%
\subsubsection{BMI Calculation}\label{bmi-calculation}}

We compute BMI categories based on the earliest records.

\includegraphics{Final_files/figure-latex/bmichart -1.pdf}
\emph{Interpret the BMI categories and their health implications.}

\hypertarget{heart-rate-analysis}{%
\subsection{Heart Rate Analysis}\label{heart-rate-analysis}}

\hypertarget{mean-and-variance-of-heart-rate}{%
\subsubsection{Mean and Variance of Heart
Rate}\label{mean-and-variance-of-heart-rate}}

We analyze heart rate data, focusing on the mean and variance.

\includegraphics{Final_files/figure-latex/heartrate-1.pdf} \emph{Discuss
the importance of heart rate monitoring in assessing cardiovascular
health.}

\hypertarget{variance-of-heart-rate}{%
\subsubsection{Variance of Heart Rate}\label{variance-of-heart-rate}}

\includegraphics{Final_files/figure-latex/heartratevariance-1.pdf}
*Elaborate on the significance of heart rate variance for health
monitoring.

\hypertarget{sleep-analysis}{%
\subsection{Sleep Analysis}\label{sleep-analysis}}

\hypertarget{sleep-patterns}{%
\subsubsection{Sleep Patterns}\label{sleep-patterns}}

We evaluate the total time in bed versus the time spent sleeping.

\includegraphics{Final_files/figure-latex/sleep-1.pdf}
\includegraphics{Final_files/figure-latex/sleep-2.pdf} \emph{Elaborate
on the impact of sleep quality on overall health and productivity.}

\hypertarget{summary-of-analysis}{%
\section{Summary of Analysis}\label{summary-of-analysis}}

\hypertarget{conclusion}{%
\subsection{Conclusion}\label{conclusion}}

*The provided dataset is not enough for a complete analysis.

*As for the analyzed user group the majority has a healthy calorie
consumption ( Calories burned ). When comparing with a lower limit of
1600 calories and upper limit of 2200 calories as suggested in
\href{https://www.healthline.com/health/fitness-exercise/how-many-calories-do-i-burn-a-day\#:~:text=You\%20burn\%20calories\%20daily\%20when,your\%20body\%20and\%20activity\%20levels.}{this
healthline article} , The majority of the participating females burn a
healthy amount of calories.

*Based on the activity level analysis of each participant, The Sedentary
time spent is significantly higher than time spent doing light Activity,
Intense activities and time spent being fairly active. The time spent
being fairly active is seen to be higher than time spend engaging in
Light Activities or being Very Active.

*The BMI graph contains the Earliest BMI's recorded for each recorded
user. This serves to give an idea of what sort of a market starts to use
wearable fitness technology. 50\% of the assessed group are overweight.
And only 30\% of the users carry a healthy BMI. If we were to assume
that the earliest recorded dates for each user indicate their earliest
days using the Fitness trackers, This is an indicator that an overweight
demographic are more inclined to use fitness technology.

*The mean heartrates for the Users indicate that the assessed user group
has healthy heartrates. All of the users have heartrates within the
healthy limit (60 - 100).

*The heartrate variance chart shows that 1/14 individuals display a
significant variance from an accepted healthy heartrate of 80 BPM.

*The sleep partterns chart indicates that most participants spend
minimal awake minutes in bed. And some users display comparatively
larger awake minutes in bed. This can be an indicator of participants
having trouble falling asleep or participants staying in bed for long
before getting off the bed. The analysis found that the average time
spent awake in bed was 8.5\%. Although this is a small number, When
comparing with sleep hours this is significant and features to help
minimize this percentage would benefit.

\hypertarget{recommendations-based-on-the-analysis}{%
\section{Recommendations based on the
Analysis}\label{recommendations-based-on-the-analysis}}

\hypertarget{future-recommendations}{%
\subsection{Future Recommendations}\label{future-recommendations}}

\begin{enumerate}
\def\labelenumi{\arabic{enumi}.}
\tightlist
\item
  \textbf{Enhanced Data Collection}: Suggest improvements in data
  collection for more robust analysis.
\item
  \textbf{User Engagement}: Encourage users to engage with their data
  regularly.
\item
  \textbf{Features to help achieve less sedentary minutes}: The
  Bellabeat app could provide reminders to take a walk etc. And help
  customers achieve better activity.
\item
  \textbf{Features to track calorie intake}: The bellabeat app can
  accomodate features to track user's calorie intake. The analysis
  indicated that the majority of the females have a healthy calorie
  consumption but it also indicated the majority to be overweight. This
  can be caused by an excess calorie intake and features to track it
  could be beneficial.
\item
  \textbf{Marketing and features catered to groups hoping to lose
  weight}: The study indicated that the majority are overweight and
  indicated that overweight individuals have a special interest in
  products similar to bellabeat. Therefore, targeted marketing and
  features beneficial to them would increase sales.
\item
  \textbf{Features to minimize awake minutes in bed}: The analysis
  indicated that there is still a significant proportion of minutes
  spent awake in bed among users. This could indicate trouble falling
  asleep or trouble getting up/ Lack of energy to get up. Bellabeat can
  implement features in their products to tackle these two scenarios.
  Such as features to help with stress management or features to track
  healthy macro and micro nutrient intakes similar to ``MyFitnessPal''
  and thereby providing suggestions to users. This would drive more
  customers to experience the benefits to real problems they face.
\end{enumerate}

\emph{In conclusion, this comprehensive analysis of Fitbit data offers
valuable insights into user behavior, ranging from daily activity
patterns to heart rate variability and sleep quality. By carefully
preprocessing, cleaning, and analyzing the data, we are better equipped
to inform Bellabeat's marketing strategy and product development. The
insights derived from this study have the potential to guide future
innovations and enhance user engagement with Bellabeat's health-focused
products. Thank you for following along on this data journey, and I look
forward to exploring further opportunities for growth and optimization.}

\textbf{Senura Charith Attanayake}

\end{document}
